\documentclass[12pt, twoside, letterpaper]{letter}
\usepackage[margin=1in]{geometry}
\usepackage[utf8]{inputenc}
\usepackage{hyperref}
\usepackage{charter}
\usepackage{microtype}

\newcommand{\PIP}{PIP\textsubscript{2}}

\signature{Alind Gupta}
\date{\small \today}
\begin{document}

\begin{letter}{Editors \\
    \textit{Current Biology}}

\opening{Dear Editors,}

We are submitting our manuscript titled
``Phosphatidylinositol 4,5-bisphosphate is essential for
cilium transition zone maturation in \textit{Drosophila melanogaster}''
to be considered for publication in \textit{Current Biology}.
In this paper, we demonstrate a novel role for the membrane lipid
phosphatidylinositol 4,5-bisphosphate, or \PIP{},
in the development of the cilium transition zone using
\textit{Drosophila melanogaster} as a model.

The cilium transition zone forms a diffusion barrier that controls
the transport of cargo entering and exiting the cilium.
As a result, it is essential to the assembly of the cilium
and its function as a sensory organelle.

In this paper, we present the following key findings:
\begin{enumerate}
\item We demonstrate that reducing \PIP{} in the Drosophila male germline
  induces the formation of longer than normal transition zones.
\item We show that the type I phosphatidylinositol phosphate kinase
  Skittles regulates transition zone length.
\item We provide evidence that hyperelongated transition zones are
  functionally defective in their ability to form a competent diffusion barrier
  and represent the outcome of an aberrant maturation process.
\item We show that the \textit{onion rings} allele of the Drosophila Exo84,
  a component of the exocyst complex, can decouple transition zone length
  with the loss of plasma membrane-cilium tethering observed in cells
  with reduced \PIP{}.
\end{enumerate}

This article is of broad interest for several reasons.
This is, to our knowledge, the first example of a positive role for the lipid
\PIP{} in cilium assembly through transition zone maturation,
particularly in an animal model.
This is in contrast to the current model
of the PIP code of cilia where \PIP{} is a disruptor of ciliary signalling.
Second, we describe an extrinsic signal for transition zone
development, which is going to be missed by proteomic approaches.
Third, we add to the current understanding of the poorly understood
process by which the transition zone matures into a functional form.
This may be important for our understanding of intermediate phenotypes
and novel modifiers of disease.

Here are the names of six possible reviewers:
\begin{enumerate}
  \item Tomer Avidor-Reiss, University of Toledo (\url{tomer.avidorreiss@utoledo.edu})
  \item
  \item
  \item
  \item
\end{enumerate}


We look forward to you favourable consideration of our manuscript.

\closing{Sincerely,}


\end{letter}
\end{document}

