\documentclass[12pt, twoside, letterpaper]{letter}
\usepackage[margin=1in]{geometry}
\usepackage[utf8]{inputenc}
\usepackage{hyperref}
\usepackage{charter}
\usepackage{microtype}

\newcommand{\PIP}{PIP\textsubscript{2}}

\signature{Julie Brill}
\date{\small \today}
\begin{document}

\begin{letter}{Editors \\
    \textit{Current Biology}}

\opening{Dear Editors,}

We are submitting our manuscript titled
``Phosphatidylinositol 4,5-bisphosphate regulates
cilium transition zone maturation in \textit{Drosophila melanogaster}''
to be considered for publication in \textit{Current Biology}.
In this paper, we demonstrate a novel role for the membrane lipid
phosphatidylinositol 4,5-bisphosphate, or \PIP{},
in the development of the cilium transition zone using
the male germline of \textit{Drosophila melanogaster} as a model.

The transition zone forms a diffusion barrier
at the base of the cilium, thereby controlling
the transport of cargo entering and exiting this organelle.
As a result, the proper formation and maturation of the transition zone
is essential for the assembly of the cilium
and its function as a sensory organelle.

In this paper, we present the following key findings:
\begin{enumerate}
\item We demonstrate that reducing \PIP{} in the Drosophila male germline
  induces the formation of longer than normal transition zones.
\item We show that the type I phosphatidylinositol phosphate kinase
  Skittles regulates transition zone length.
\item We provide evidence that hyperelongated transition zones are
  functionally defective.
\item We show that the \textit{onion rings} allele of Drosophila Exo84,
  a component of the exocyst complex, can decouple transition zone length
  from the loss of plasma membrane-cilium tethering observed in cells
  with reduced levels of \PIP{}.
\end{enumerate}

This article is of broad interest for several reasons.
First, we show that the poorly understood process of
transition zone maturation, by which a nascent transition zone
matures into a fully functional one,
is regulated by the lipid \PIP{}, which is known to be enriched
at the base of cilia from multiple studies.
Second, we show that the Drosophila type I phosphatidylinositol
phosphate kinase Skittles regulates transition zone
maturation, and that it does so \textit{in situ},
providing further evidence that these enzymes may be genetic
modifiers of diseases of cilia.
Third, we show that the exocyst, which has been shown to be important for
cilium formation in multiple studies, regulates cilium maintenance
by enabling plasma membrane-cilium attachment but not transition zone maturation.

Here are the names of six possible reviewers:
\begin{enumerate}
  \item B{\'e}n{\'e}dicte Durand, Université Claude Bernard Lyon-1 (\url{benedicte.durand@univ-lyon1.fr})
  \item Daniel Eberl, University of Iowa (\url{daniel-eberl@uiowa.edu})
  \item Yukiko Yamashita, University of Michigan (\url{yukikomy@umich.edu})
  \item Timothy Megraw, Florida State University (\url{timothy.megraw@med.fsu.edu})
  \item Tomer Avidor-Reiss, University of Toledo (\url{tomer.avidorreiss@utoledo.edu})
    \item Andrew Jarman, University of Edinburgh (\url{andrew.jarman@ed.ac.uk})
\end{enumerate}

We look forward to your favourable consideration of our manuscript.

\closing{Sincerely,}


\end{letter}
\end{document}
