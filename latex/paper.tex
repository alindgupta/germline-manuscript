\documentclass[12pt, twoside, letterpaper]{article}
\usepackage[margin=0.8in]{geometry}
\usepackage[parfill]{parskip}
\usepackage[utf8]{inputenc}
\usepackage[T1]{fontenc}
\usepackage{authblk}
\usepackage{charter}
\usepackage[numbers]{natbib}
\usepackage{graphicx}
\graphicspath{{../figures/}}
\usepackage{enumitem}
\usepackage{hyperref}
\usepackage{titlesec}
\usepackage{lineno}
\usepackage{setspace}

% bibliography style
\bibliographystyle{unsrtcurrbiol}

% title font size
\titleformat*{\section}{\large\bfseries}
\titleformat*{\subsection}{\normalsize\bfseries}

% line spacing
\renewcommand{\baselinestretch}{1.1}

% newcommand
\newcommand{\PIP}{PIP\textsubscript{2}}
\newcommand{\sigd}{$\beta$\textit{2t}-SigD}
\newcommand{\sktl}{\textit{sktl}}

% renewcommand
\renewcommand*{\Affilfont}{\footnotesize \normalfont}
\renewcommand*{\Authfont}{\normalsize}


\title{\vspace{-1cm} \bfseries\large Phosphatidylinositol 4,5-bisphosphate regulates cilium transition zone maturation in \textit{Drosophila melanogaster}}
\author[1,2]{Alind Gupta \thanks{\url{alind.gupta@mail.utoronto.ca}}}
\author[1,2]{Julie A. Brill \thanks{\url{julie.brill@sickkids.ca} (Corresponding author)}}

\affil[1]{\small Department of Molecular Genetics, University of Toronto, Toronto, ON M5S 1A8, Canada}
\affil[2]{\small Program in Cell Biology, Hospital for Sick Children, Toronto, ON M5G 1X8, Canada}

\date{\small \today}


\begin{document}

\maketitle

\begin{doublespacing}
  \begin{linenumbers}
    
    \section{Summary}
    Cilia are sensory organelles that are essential for human development
    and physiology \citep{satir2010primary}.
    They function primarily by relaying extracellular stimuli to cells
    \citep{satir2010primary},
    but can also act as motile propellers in certain cell types
    \citep{bloodgood2010sensory}.
    A large number of genetic disorders linked to cilium dysfunction are
    associated with proteins that localize to the cilium transition zone (TZ),
    a structure at the base of cilia that regulates trafficking in and out
    of the cilium \citep{reiter2012base, szymanska2012transition}.
    Despite numerous efforts to elucidate TZ proteins and their functions,
    processes underlying maturation of TZs are not well understood.
    We report a role for the membrane lipid phosphatidylinositol 4,5-bisphosphate
    (\PIP{}) in TZ maturation and length regulation
    in the \textit{Drosophila melanogaster} male germline.
    Reduction of cellular \PIP{} levels by expression of the
    \textit{Salmonella} phosphoinositide phosphatase SigD or mutation of
    the type I phosphatidylinositol phosphate kinase (PIPKI) Skittles (Sktl)
    induced hyperelongation of TZs after they anchor at the plasma membrane.
    Hyperelongated TZs exhibit signs of functional defects, such as the inability
    to migrate and loss of membrane tethering.
    We also report that the Exo84 mutant \textit{onion rings} (\textit{onr})
    decouples TZ hyperelongation from loss of cilium-plasma membrane tethering.
    Our results reveal a requirement for \PIP{} in supporting ciliogenesis
    through TZ maintenance.
    
    \section{Results and Discussion}
    Cilia are sensory organelles present on almost all cells in the human body
    \citep{satir2010primary}.
    They are important for transducing signalling pathways in response to extracellular
    cues.
    Consistent with their importance in cell signalling,
    defects in cilium formation are associated with genetic disorders
    known collectively as ciliopathies, which can display
    neurological defects \citep{valente2014primary},
    skeletal abnormalities \citep{hammarsjo2017novel},
    and infertility \citep{inaba2016sperm} in addition to other phenotypes
    \citep{waters2011ciliopathies}.
    Many ciliopathies are associated with mutations in proteins that localize
    to the cilium transition zone (TZ), the proximal-most region of the cilium that
    functions as a diffusion barrier and regulates the
    bi-directional transport of protein and lipid cargo at the base of the cilium
    \citep{reiter2012base, szymanska2012transition}.
    The TZ is essential for cilium formation (ciliogenesis) and function.
    The conserved TZ protein CEP290, for example, is mutated in at least
    six different ciliopathies with a spectrum of clinical features and severities
    , and is important for cilium formation
    and function in both humans \citep{shimada2017vitro, stowe2012centriolar}
    and Drosophila \citep{basiri2014migrating}.
    Although the protein composition of TZs has been investigated in various
    studies ranging from single gene analysis to proteomics,
    processes underlying TZ maturation are relatively understudied.

    Ciliogenesis begins with the assembly of TZ proteins at the tips of
    the basal body.
    This nascent TZ undergoes maturation, during which its structure and protein
    composition changes, allowing for anchoring of the TZ to the membrane and
    establishment of a compartmentalized space bounded by the ciliary membrane
    and the TZ.
    In the Drosophila male germline, nascent TZs first assemble on basal bodies
    during early G2 phase in spermatocytes and anchor to the plasma membrane.
    This is followed by anchoring of cilia to the plasma membrane,
    rearrangment of microtubules in the axoneme \citep{gottardo2013cilium},
    and establishment of a
    ciliary membrane that persists through meiosis \citep{riparbelli2012assembly}.
    TZ maturation has been described in
    \textit{Paramecium} \citep{aubusson2015transition}
    \textit{Caenorrhabditis elegans} \citep{serwas2017centrioles} and
    \textit{Drosophila melanogaster} \citep{gottardo2013cilium}
    and is most readily observed by an increase in TZ length
    in the Drosophila male germline.

    Here, we show that the membrane lipid phosphatidylinositol 4,5-bisphosphate (\PIP{})
    is essential for proper TZ maturation in developing male germ cells
    in \textit{D. melanogaster}.
    \PIP{} is relatively enriched within the plasma membrane and the base of cilia
    in the male germline.
    We show that expression of the bacterial PIP phosphatase SigD
    or inactivation of the PIP 5-kinase Skittles (Sktl) that localizes to cilia
    induces longer than normal TZs without affecting TZ assembly.
    These hyperelongated TZs show hallmarks of aberrant function similar to those
    found in previously studied TZ mutants,
    including loss of cilium-membrane associations.
    We also find that the \textit{onion rings} allele of the
    ciliopathy-associated \textit{EXOC8} gene encoding the human Exo84
    decouples TZ hyperelongation from loss of cilium-membrane contacts.
    Our results suggest that the plasma membrane lipid \PIP{} regulates TZ length,
    maturation and function, thereby acting to support ciliogenesis.

    \subsection{\PIP{} is essential for transition zone maturation}
    To investigate whether reduction of cellular \PIP{} affects ciliogenesis in the
    Drosophila male germline,
    we used transgenic flies expressing the Salmonella phosphoinositide phosphatase SigD
    under the $\beta$2-tubulin promoter (referred to herein as \sigd{}),
    which is expressed exclusively in spermatocytes
    \citep{wei2008depletion, fabian2010phosphatidylinositol}.
    We previously showed that \sigd{} induces male sterility
    due in part to defects in flagellar biosynthesis \citep{wei2008depletion}.
    To examine whether this may be caused by aberrant TZ function,
    we examined the localization of fluorescently-labelled
    centriolar/basal body protein
    Ana1 (CEP295 homolog) \citep{goshima2007genes, blachon2009proximal}
    and the TZ protein Cep290 \citep{basiri2014migrating}
    during early steps of cilium assembly in spermatocytes in
    wild-type controls and \sigd{}.
    As shown in Figure 1,
    while Cep290 on nascent TZs look similar in wild-type and \sigd{}, 
    it grew to significantly longer lengths in \sigd{}
    compared to wild-type controls.
    Ana1 lengths were not affected, and we did not observe an inverse
    correlation between Cep290 and Ana1 length.
    This hyperelongation phenotype was not specific to Cep290;
    all TZ markers examined showed hyperelongation, including
    acetylated tubulin \citep{gottardo2013cilium},
    Chibby \citep{enjolras2012drosophila} and
    MKS \citep{slaats2015mks1, vieillard2016transition} proteins (Figure 1D).
    TZ hyperelongation was a highly penetrant phenotype (\textgreater 90\%)
    and showed high intra-cyst correlation (\textgreater 0.95) suggestive
    of a dosage-based response to a shared cellular factor,
    presumably SigD.
    Hyperelongated TZs persisted through meiosis.
    Despite TZ hyperlongation, however, axonemes were able to
    elongate post-meiosis (Figure 1E).
    As we showed previously, the ultrastructure of these axonemes
    was frequently aberrant,
    either lacking symmetry completely or containing triplet microtubules
    instead of doublets \citep{wei2008depletion}.

    \subsection{The type I PIP kinase Skittles localizes to centrioles and regulates TZ length}
    While \PIP{} is the major substrate for SigD in eukaryotic cells \textit{in vivo}
    \citep{terebiznik2002elimination, zhou2001salmonella, sengupta2013depletion},
    SigD can also dephosphorylate other PIPs \textit{in vitro}
    \citep{norris1998sopb}.
    To address whether TZ hyperelongation observed in \sigd{} represented a
    physiologically relevant phenotype in response to decreased \PIP{},
    we attempted to rescue it by
    co-expressing fluorescently-tagged full-length Sktl (Sktl) with SigD
    under the same promoter.
    We found that Sktl expression was able to suppress TZ elongation
    in a cilium-autonomous manner (Figure 2A and B).
    We also found that \textit{sktl} clones using Flp/FRT-based mitotic recombination
    showed TZ hyperelongation (Figure 2C).

    Since Sktl expression suppressed TZ hyperelongation,
    we investigated whether Sktl could localize to centrioles or cilia
    in the Drosophila germline.
    Similar to the human PIPKI$\gamma$ isoform \citep{xu2014pipkigamma},
    we found that Sktl localizes to basal bodies in the Drosophila male germline.
    As shown in Figure 2, Drosophila PIPKIs arose from the recent duplication of
    the ancestral PIPKI gene,
    and are not orthologous to specific vertebrate PIPKI isoforms.
    Sktl has diverged more than its paralog PIP5K59B, and seems
    to be functionally related to PIPKI$\gamma$ in its roles at cilia,
    as well as the \textit{C. elegans} PPK-1 \citep{xu2014pipkigamma}.
    However, unlike the human PIPKI$\gamma$, which functions to license TZ assembly
    by promoting CP110 removal from basal bodies \citep{xu2016phosphatidylinositol},
    our results suggest that Sktl functions in
    regulating TZ length in the Drosophila male germline
    but not its assembly.
    Cp110 is dispensable for cilium formation in Drosophila \citep{franz2013cp110}
    and we found its localization to be unaffected in \sigd{} (Supplementary figure 1).

    \subsection{Hyperelongated transition zones exhibit signs of functional defects}
    While shorter spermatocyte TZs have been reported in mutants of TZ proteins in Drosophila
    \citep{vieillard2016transition, pratt2016drosophila},
    no genetic manipulation has been shown to induce hyperelongated TZs.
    We sought to examine whether TZ hyperelongation due to SigD expression
    affected TZ function.
    During germ cell elongation following meiosis, the TZ detaches from
    the basal body and migrates along the growing axoneme, maintaining a ciliary compartment
    at the distal-most {\raise.17ex\hbox{$\scriptstyle\sim$}}5$\mu$m where tubulin
    is presumably incorporated into the axoneme.
    We found that
    TZs that assembled in \sigd{} cells are non-migratory despite the growth of the
    axoneme and the cell membrane, as shown by Unc and Cep290 localization.
    We previously described ``comet-shaped'' localization of Unc-GFP in these cells,
    and we find that these hyperelongated Unc signals persist into cellular elongation
    despite elongation of the axoneme (Figure 3).

    We previously described the presence of occasional triplet microtubules
    in electron micrographs of axonemes from \sigd{} \citep{wei2008depletion}.
    While centrioles and basal bodies contain microtubule triplets,
    restriction of the C-tubule at the TZ causes the axoneme to
    consist of microtubule doublets rather than triplets throughout its length.
    Consistent with a defect in TZ barrier formation and presence of
    microtubule triplets in axonemes,
    we observed that a subset of cilia in \sigd{}
    contained puncta of Ana1, a core centriolar protein, at the tips
    of TZs marked with Cep290.
    This TZ-distal Ana1 localization is observed at a relatively low penetrance
    and high intra-cyst correlation.
    We found that treatment of germ cells with Taxol as in \citep{riparbelli2013unique}
    could increase the penetrance of this phenotype.
    Taxol treatment stabilizes microtubules and is thought to disrupt TZ maturation
    by interrupting microtubule rearrangements, and similarly induces much longer cilia
    \citep{riparbelli2012assembly}.
    Taxol treatment alone did not induce this phenotype in wild-type controls in our hands
    at varying concentrations ($p$ <0.01 with 5\% penetrance),
    suggesting that loss of \PIP{} acts upstream of or prior to microtubule rearrangement.
    Triplet microtubules within the axoneme are also observed in Taxol-treated cells \citep{riparbelli2013unique}.
    It is possible that the TZ defect allows C-tubules and some centriolar proteins
    to migrate to the distal end.
    We did not observe this behaviour with Asterless (CEP152)
    \citep{dzhindzhev2010asterless, blachon2008drosophila} ($p$ <0.01),
    another centriolar protein, suggesting that these TZ-distal
    sites are not fully centriolar in protein composition.

    \subsection{The \textit{onion rings} mutant decouples defects found in cells with reduced levels of \PIP{}}
    We previously showed that male flies homozygous for the \textit{onion rings}
    (\textit{onr}) mutant of Drosophila Exo84 were sterile, with phenotypes
    similar to \sigd{} \citep{wei2008depletion}.
    Exo84 is a component of the octameric exocyst complex, which binds
    \PIP{} and regulates membrane trafficking at the plasma membrane.
    To investigate whether defects in TZ hyperelongation could be explained by
    defective Exo84 function, we examined TZs in \textit{onr} mutants.
    We found that the \textit{onr} mutant did not display hyperelongated TZs (Figure 5),
    suggesting that Exo84 is dispensable for TZ maturation.

    Due to the involvement of exocyst in membrane trafficking at the plasma membrane,
    we examined whether cilium-associated membrane dynamics were affected in \sigd{}
    or \textit{onr} mutants similar to \textit{dila; cby} mutants
    \citep{vieillard2016transition}.
    We found that while TZs in \sigd{} and \textit{onr} cells anchored properly
    to the plasma membrane, they were unable to either establish or maintain
    membrane connections, and were rendered cytoplasmic (Figure 5) similar to
    \textit{dila; cby} mutants.
    Although we were not able to determine the localization of Exo84 due to the
    lack of available reagents,
    we found that Exo70 localized to centrioles (Figure 4D),
    but not Sec6 (Supplementary figure 2).
    Exo70 can directly bind to \PIP{} \citep{he2007exo70}.
    Our results suggest that the exocyst may act downstream of \PIP{} to
    enhance cilium-membrane associations, and that TZ hyperelongation and loss of
    membrane-cilium association are genetically separable phenotypes.


    The initial steps of ciliogenesis involve the assembly of a nascent
    TZ on the basal body.
    This nascent TZ ``matures'' through sequential assembly of additional
    TZ proteins into a functional TZ capable of acting as a diffusion barrier
    and supporting axoneme elongation and ciliary signalling.
    In the Drosophila male germline, TZ maturation occurs during the
    extended G2 phase in spermatocytes prior to meiosis,
    and is readily observed by an increase in TZ length
    and various microtubule rearrangements.
    Axoneme elongation occurs in telophase II following meiosis in
    spermatids, where a functional TZ detaches from the basal body
    and migrates along the growing axoneme to compartmentalize the distal tip
    in Cep290-dependant manner \citep{basiri2014migrating}.

    We found that reduction of \PIP{} levels in the male germline by expression
    of the lipid phosphatase SigD or mutation in the type I PIP kinase
    Skittles (Sktl) lead to much longer TZs by the end of spermatocyte G2 phase.
    Presumably, loss of \PIP{} abolishes a signal to stop TZ lengthening during
    a particular step in maturation.
    We were able rescue TZ hyperelongation by expression of full-length Sktl to
    a large degree.
    Interestingly, some cells showed cilium-autonomous suppression of lengths.
    Development of the Drosophila male germline occurs in syncytial ``cysts''
    of cells which share the same cytoplasm.
    Therefore, our results suggest that Skittles might function \textit{in situ}
    to regulate TZ maturation in these cells.


    In humans, \textit{PIPKI$\gamma$} is linked to lethal congenital contractural
    syndrome type 3 (LCCS3), which has been suggested to represent a ciliopathy.
    The recent discovery of the role of LCCS1-associated GLE3 protein in cilia
    further shows this.


    Members of the exocyst complex, such as Sec10 and Sec6, have been shown
    to be important for cilium formation in cultured cell lines and zebrafish
    \citep{zuo2009exocyst, lobo2017exocyst, seixas2016arl13b}.
    The exocyst is known to bind to \PIP{}
    \citep{he2007exo70, zhang2008membrane}, and we previously described that
    the \textit{onr} allele of Drosophila \textit{Exo84} phenocopied
    male germline defects observed in \sigd{} \citep{fabian2010phosphatidylinositol}.
    In this paper, we show that the \textit{onr} allele of the Drosophila Exo84
    phenocopies the loss of cilium-membrane contacts in \sigd{},
    but not TZ hyperelongation.
    Proper TZ function is required for membrane tethering, as \textit{dila; cby}
    mutants, which contain abnormal TZs also show cytoplasmic cilia similar to
    \sigd{} \citep{vieillard2016transition}.
    Our results suggest that the exocyst is required after the initial stages
    of TZ assembly and maturation for TZ function.

    \section{Methods}
    \subsection{Transgenic flies}
    Drosophila stocks were cultured on cornmeal molasses agar medium at 25$^{\circ}$C
    and 50\% humidity.
    Stocks expressing $\beta$\textit{2t}::\textit{Styp}\textbackslash{SigD} (chromosome \textit{3})
    and $\beta$\textit{2t}::YFP-Sktl (chromosome \textit{2})
    were described previously \citep{wei2008depletion, fabian2010phosphatidylinositol}.
    $\beta$\textit{2t}::mCherry-Sktl was made as in \citep{wei2008depletion},
    and a \textit{P}-element insertion on chromosome \textit{2} was used for rescue using the low-level
    expression vector \textit{tv3} \citep{wong2005pip2}.
    Ana1-tdTomato (chromosome \textit{2}) and Cep290-GFP (chromosome \textit{3}) were provided
    by T. Avidor-Reiss \citep{basiri2014migrating}.
    \textit{Sp}/\textit{CyO}; Unc-GFP was originally provided by M. Kernan \citep{baker2004mechanosensory}.
    Stocks expressing GFP-tagged Chibby and Mks1 were
    provided by B. Durand \citep{enjolras2012drosophila, vieillard2016transition}.
    The \textit{onr} mutant was described previously \citep{giansanti2015exocyst}.
    Stocks for generating \textit{sktl}\textsuperscript{2.3} clones were originally provided by
    A. Guichet \citep{gervais2008pip5k}.
    \textit{w}\textsuperscript{1118} was used as the wild-type control.

    \subsection{Antibodies}
    The following primary antibodies were used for immunofluorescence
    at the indicated concentrations -
    rabbit anti-Tulp (gift from W. McGinnis \citep{ronshaugen2002structure}), 1:500;
    chicken anti-GFP IgY (abcam) 1:1000;
    rat anti-RFP IgG (5F8, ChromoTek), 1:1000;
    rabbit anti-Sktl (courtesy of P. Raghu), 1:500;
    rabbit anti-\textit{d}Sec6 (courtesy of U. Tepass), 1:500;
    mouse anti-acetylated $\alpha$-tubulin 6-11-B (Sigma-Aldrich), 1:1000.
    Secondary antibodies were Alexa 488- and Alexa 568-conjugated
    anti-mouse, anti-rabbit and anti-chicken
    IgG (Molecular Probes) at 1:1000.
    DAPI at 1:1000 was used to label chromatin.

    \subsection{Fluorescence microscopy}
    For live imaging, testes were dissected in phosphate bufferd saline (PBS).
    If staining for chromatin, intact testes were incubated in PBS with
    Hoechst (1:5000) for 5 minutes at this point.
    Testes were transferred to a polylysine-coated glass slide in a drop of PBS,
    ruptured using a syringe needle and
    squashed under a glass coverslip using Kimwipes.
    The edges of the coverslip were sealed with nail polish
    and the specimen was visualized immediately using an epifluorescence microscope.

    For Taxol treatments, testes from larvae or pupae expressing
    Ana1-tdTomato; Cep290-GFP were
    dissected into Shields and Sang M3 medium (Sigma-Aldrich) supplemented
    with a predefined
    concentration of Taxol and incubated overnight in a humidified sterile
    chamber in the dark at room temperature. They were then squashed
    in PBS and imaged live.

    For CellMask staining, cells were spilled from testes in M3 medium onto
    a sterilized glass-bottom dish that was pre-treated with sterile polylysine solution
    to enable cells to adhere.
    CellMask Deep Red (Invitrogen) solution (20 $\mu$g/mL) was added to the medium dropwise
    immediately before visualization under a confocal microscope.

    For immunochemistry, testes were dissected in PBS,
    transferred to a polylysine-coated glass slide in a drop of PBS,
    ruptured with a needle, squashed and frozen in liquid nitrogen for 5 minutes.
    The slides were transferred to ice-cold methanol for 5-10 minutes for pre-fixation.
    For staining with the anti-Sktl antibody, this pre-fixation step
    was omitted to preserve cellular membranes.
    Samples were fixed in 4\% paraformaldehyde in PBS,
    permeabilized and blocked in PBS with 0.1\% Triton-X and 0.3\% bovine
    serum albumin and incubated with primary antibodies overnight at 4$^{\circ}$C.
    followed by three 5-minute washes with PBS and 1 hour incubation
    with secondary antibodies and three 5-minute washes with PBS.
    Samples were mounted in Dako (Agilent) and imaged with
    a Leica widefield microscope
    or Nikon A1R scanning confocal microscope (SickKids imaging facility).

    \subsection{Statistical methods}
    Statistical analysis and graphing was performed using R (version 3.4)
    \citep{r}.
    A Gaussian jitter was applied when plotting
    results in Figures 1 and 2 for clearer visualization of trends,
    but raw data was used for all analysis.
    Statistical tests for ``absence of phenotype'' were computed under a
    hypothesis assuming binomial likelihood (i.e. assuming a fixed probability of the
    phenotype occurring).
    All $t$-tests were unpaired and two-sided.
    A significance level of 0.01 was fixed \textit{a priori} for all classical analysis.
    All raw data and code for analysis and plotting can be found online
    at \url{http://www.github.com/alindgupta/germline-paper/}.

    \subsection{Phylogenetic analysis}
    Candidate orthologs of Skittles or PIP5K9B were queried
    from Inparanoid and FlyBase (version 8.0).
    Poorly annotated protein sequences were confirmed
    to encode type I phosphatidylinositol phosphate
    kinases using reciprocal BLAST search.
    Phylogeny.fr (\url{http://www.phylogeny.fr}) was used for
    phylogenetic reconstruction with T-Coffee for multiple alignment
    and MrBayes for tree construction.
    The output was converted to a vector image in Illustrator
    and colours were added for the purpose of illustration.

    \section{Supplemental information}
    Includes two figures.
    
    \section{Acknowledgements}
    We are grateful to Brian Ciruna for insightful comments on the project,
    and Bénédicte Durand for providing (at the time of experiments)
    unpublished fly stocks.
    This work was supported by a grant from the Canadian Institute for
    Health and Research to J.A.B (Placeholder for Grant number).
    A.G. was supported by an Open Fellowship and Ontario Graduate Scholarship.
    
    \section{Author contributions}
    A.G. and J.A.B. conceived the project.
    A.G. performed all the experiments and analysis, and wrote the manuscript.
    J.A.B. supervised the project.
    
    \section{Declaration of interests}
    The authors declare no competing interests.

    \section{Figures}
    
    \textbf{Figure 1. SigD expression induces transition zone hyperelongation.}
    \begin{enumerate}[label={(\Alph*)}, nolistsep]
    \item Schematic diagram of the stages of ciliogenesis in the Drosophila male germline.
      Stages in brackets correspond to those in \citep{cenci1994chromatin}.
    \item \sigd{} expression induces Cep290 hyperelongation in late G2 cilia.
    \item Paired Ana1-Cep290 lengths in early and late G2 phase in spermatocytes.
    \item \sigd{} expression induces hyperelongation of Chibby and Mks1 in late G2 phase.
    \item TZ hyperelongation persists through meiosis but does not affect
      axoneme outgrowth.
    \end{enumerate}
    
    \textbf{Figure 2. Sktl is important for transition zone maturation.}
    \begin{enumerate}[label={(\Alph*)}, nolistsep]
    \item Expression of full-length Sktl suppresses \sigd{}-induced TZ hyperelongation.
    \item Quantification of Cep290 and Ana1 lengths from control, \sigd{} and Sktl; \sigd{} ($n$\textgreater 60) from (A).
    \item \textit{sktl}$^{2.3}$ clones exhibit TZ hyperelongation. Cilia are marked using
      Unc-GFP.
    \item Phylogenetic tree of PIP5Ks showing conservation of cilium-associated functions.
      The scale bar represents expected amino acid substitutions per site and branch support values are shown in red (a value of 1 indicates maximum support).
      Abbreviations: Cele (\textit{Caenorhabditis elegans}), Spur (\textit{Strongylocentrotus purpuratus}), Amel (\textit{Apis mellifera}), Aaeg (\textit{Aedes aegypti}), Dana (\textit{Drosophila ananassae}), Dmel (\textit{Drosophila melanogaster}), Hsap (\textit{Homo sapiens}), Mmus (\textit{Mus musculus}), Xtro (\textit{Xenopus tropicalis}), Cint (\textit{Ciona intestinalis}), Scer (\textit{Saccharomyces cerevisiae}).
    \end{enumerate}
    
    
    \textbf{Figure 3. Hyperelongated transition zones display functional defects.}
    \begin{enumerate}[label={(\Alph*)}, nolistsep]
    \item 
    \item
    \item
    \item
    \end{enumerate}


    \textbf{Figure 4. The \textit{onion rings} (\textit{onr}) allele of \textit{d}Exo84 decouples TZ hyperlongation from loss of plasma membrane contacts.}
    \begin{enumerate}[label={(\Alph*)}, nolistsep]
    \item Cells expressing \sigd{} do not maintain plasma membrane-cilium tethering. The plasma membrane is marked with CellMask.
    \item \textit{onr} mutants do not maintain plasma membrane-cilium tethering.
    \item Cells mutant for \textit{onr} do not display hyperelongated acetylated tubulin signal at the cilium.
    \item GFP-tagged Exo70 localizes to basal bodies.
    \end{enumerate}
    
  \end{linenumbers}
\end{doublespacing}

\bibliography{bibttc}

%% 
% FIGURE 1
% 
% \textbf{Figure 1. SigD expression induces transition zone hyperelongation.}
% \begin{enumerate}[label={(\Alph*.)}, nolistsep]
% \item Schematic diagram of the stages of ciliogenesis in the Drosophila male germline.
%   Stages in brackets correspond to those from \citep{cenci1994chromatin}.
% \item \sigd{} expression induces Cep290 hyperelongation in late G2 cilia.
% \item Paired Ana1-Cep290 lengths in early and late G2 phase in spermatocytes.
%   Numbers in red represent correlation between Cep290 and Ana1 lengths.
% \item \sigd{} expression induces hyperelongation of Chibby and Mks1.
% \item TZ hyperelongation persists through meiosis but does not affect
%   axoneme outgrowth.
% \end{enumerate}
% 
% \begin{figure}[ht]
%   \includegraphics[scale=0.9]{figure-1/figure1.png}
% \end{figure}
% \newpage
% 
% %% 
% %%%% 
% %%% FIGURE 2
% %%% 
% \textbf{Figure 2. Sktl is important for transition zone maturation and localizes to basal bodies.}
% \begin{enumerate}[label={(\Alph*.)}, nolistsep]
% \item Expression of full-length Sktl suppresses \sigd{}-induced TZ hyperelongation.
% \item Quantification of Cep290 and Ana1 lengths from control, \sigd{} and Sktl; \sigd{} ($n$\textgreater 60) from (A).
% \item \textit{sktl}$^{2.3}$ clones exhibit TZ hyperelongation. Cilia are marked using
%   Unc-GFP.
% \item Phylogenetic tree of PIP5Ks showing conservation of cilium-associated functions.
%   The scale bar represents expected amino acid substitutions per site and branch support values are shown in red (a value of 1 indicates maximum support).
%   Abbreviations: Cele (\textit{Caenorhabditis elegans}), Spur (\textit{Strongylocentrotus purpuratus}), Amel (\textit{Apis mellifera}), Aaeg (\textit{Aedes aegypti}), Dana (\textit{Drosophila ananassae}), Dmel (\textit{Drosophila melanogaster}), Hsap (\textit{Homo sapiens}), Mmus (\textit{Mus musculus}), Xtro (\textit{Xenopus tropicalis}), Cint (\textit{Ciona intestinalis}), Scer (\textit{Saccharomyces cerevisiae}).
% \end{enumerate}
% 
% \begin{figure}[ht]
%   \includegraphics[scale=0.9]{figure-2/figure2.png}
% \end{figure}
% \newpage
% 
% 
% %% 
% % FIGURE 3
% % 
% \textbf{Figure 3. Hyperelongated transition zones display functional defects.}
% \begin{enumerate}[label={(\Alph*)}, nolistsep]
% \item
% \item
% \item
% \item
% \end{enumerate}
% 
% \begin{figure}[ht]
%   \includegraphics[scale=0.8]{figure-3/figure3.png}
% \end{figure}
% \newpage
% 
% 
% %%% 
% %% FIGURE 4
% %% 
% \textbf{Figure 4. The \textit{onion rings} (\textit{onr}) allele of \textit{d}Exo84 decouples TZ hyperlongation from loss of plasma membrane contacts.}
% \begin{enumerate}[label={(\Alph*)}, nolistsep]
% \item Cells mutant for \textit{onr} do not display hyperelongated TZs marked with acetylated tubulin.
% \item Quantification from (A).
% \item Cells mutant for \textit{onr} display loss of plasma membrane tethering.
% \end{enumerate}
% 
% \begin{figure}[ht]
%   \includegraphics[scale=0.9]{figure-4/figure4.png}
% \end{figure}
% \newpage


\end{document}

%%% Local Variables:
%%% mode: latex
%%% TeX-master: t
%%% End
