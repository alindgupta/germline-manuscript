\documentclass[12pt, twoside, letterpaper]{article}
\usepackage[margin=0.7in]{geometry}
\usepackage[parfill]{parskip}
\usepackage[utf8]{inputenc}
\usepackage[T1]{fontenc}
\usepackage{authblk}
\usepackage{charter}
\usepackage[numbers]{natbib}
\usepackage{graphicx}
\usepackage{hyperref}
\usepackage{titlesec}

\bibliographystyle{unsrtapalike}


% title font size
\titleformat*{\section}{\normalsize\bfseries}
\titleformat*{\subsection}{\normalsize\bfseries}

% line spacing
\renewcommand{\baselinestretch}{1.1}

% new commands
\newcommand{\PIP}{PIP\textsubscript{2}}
\newcommand{\sigd}{$\beta$\textit{2tub}-SigD}
\newcommand{\sktl}{\textit{sktl}}
\renewcommand*{\Affilfont}{\footnotesize \normalfont}
\renewcommand*{\Authfont}{\normalsize}


\title{\bfseries\large Phosphatidylinositol 4,5-bisphosphate regulates cilium transition zone maturation in Drosophila}
\author[1,2]{Alind Gupta}
\author[1,2]{Julie A. Brill}
\affil[1]{Department of Molecular Genetics, University of Toronto, Toronto, ON M5S 1A8, Canada}
\affil[2]{Program in Cell Biology, Hospital for Sick Children, Toronto, ON M5G 1X8, Canada}
\date{}


\begin{document}
\maketitle
\raggedright
\begin{abstract}
  \vspace*{-0.7em}
  Cilia are sensory organelles that are important for human development
  and physiology.
  A large number of genetic disorders affecting cilia are associated with
  mutations in proteins that localize to the transition zone (TZ),
  a structure at the base of cilia required for ciliogenesis and
  regulation of trafficking in and out of the cilium.
  Here, we report that the membrane lipid phosphatidylinositol
  4,5-bisphosphate(\PIP{}) is essential for proper maturation and function
  of TZs in the male \textit{Drosophila melanogaster} germline.
  Reduction of cellular \PIP{} levels by expression of the
  \textit{Salmonella} phosphoinositide phosphatase SigD or mutation of
  the type I phosphatidylinositol phosphate kinase Skittles (Sktl)
  induces hyperelongation of TZs following their docking at the plasma
  membrane.
  We provide evidence showing that these hyperelongated TZs are functionally
  defective, and that the Exo84 mutant \textit{onion rings} decouples the
  transition zone hyperelongation from the role for \PIP{} in cilium-membrane
  tethering.
  Our observations reveals a requirement for \PIP{} at the base of cilia
  in regulating TZ length and function.
\end{abstract}

\section{Introduction}
Cilia are sensory organelles present on almost all cells in the human body
\citep{satir2010primary}.
They function primarily by providing sensory inputs to cells, but can
also function as motile propellers in certain cells
\citep{bloodgood2010sensory}.
Consistent with their pervasiveness and importance in cell signalling,
defects in cilium formation are associated with various genetic disorders
known collectively as ciliopathies, which can manifest as
neurological disorders \citep{valente2014primary},
skeletal abnormalities \citep{hammarsjo2017novel, waters2011ciliopathies}
and infertility \citep{inaba2016sperm}.

A large number of ciliopathies are linked to mutations in proteins that
localize to the cilium transition zone (referred to herein as TZ).
The TZ is the proximal-most compartment of the cilium which
acts as a diffusion barrier that regulates the
bi-directional transport of protein and lipid cargo at the base of the cilium,
and is essential for cilium formation and function in cell signalling.
The conserved TZ protein CEP290, for example, is mutated in at least
five distinct ciliopathies, and is essential for cilium formation in both humans
and Drosophila.
Although the composition of TZs is well understood, processes underlying
transition zone maturation and cell membrane anchoring are not as well-studied.
In particular, why mutations in the same TZ protein can give rise to a plethora
of phenotypic outputs is less understood, and may point at modifier mutations
or mutations that are not as severe as to entirely abolish ciliogenesis.

In the male Drosophila germline, nascent TZs first assemble on basal bodies
during early pre-meiotic G2 phase (spermatocytes), anchor directly to the plasma membrane.
This is followed by establishment of cilium-membrane connections and a
ciliary membrane that persists during meiosis \citep{riparbelli2012assembly}
through a poorly understood mechanism
involving microtubule rearrangements \citep{gottardo2013cilium}.
Ciliogenesis begins with the assembly of TZ proteins at the tips of
the nascent basal body.
This is followed by the anchoring of the TZ to the membrane and
establishment of a compartmentalized space bounded by the ciliary membrane
and the TZ.
TZ maturation has been described in
\textit{Paramecium} \citep{aubusson2015transition}
\textit{Caenorrhabditis elegans} \citep{serwas2017centrioles} and
\textit{D. melanogaster} \citep{gottardo2013cilium}
and involves TZ elongation.
Possibly, this is similar to the initial steps of TZ assembly and reorganization
in well-studied vertebrate cilia.

Here, we show that the lipid phosphatidylinositol 4,5-bisphosphate (\PIP{})
is essential for proper TZ maturation in developing \textit{Drosophila melanogaster}
male germ cells.
Reduction of cellular \PIP{} by expression of the bacterial PIP phosphatase SigD
or by inactivation of the PIP 5-kinase Skittles (Sktl) that localizes to cilia
induces longer than normal TZs.
These hyperelongated TZs show hallmarks of aberrant function similar to those
found in previously studied TZ mutants, including loss of cilium-membrane associations.
We also find that the \textit{onion rings} allele of
ciliopathy-associated exocyst protein Exo84/EXOC8 decouples TZ hyperelongation
from loss of cilium-membrane contacts.
Our results suggest that the plasma membrane lipid \PIP{} regulates TZ length,
maturation and function, acting to support ciliogenesis.


\section{Results}

\subsection{\PIP{} is essential for transition zone maturation}
To investigate whether reduction of cellular \PIP{} affects ciliogenesis in the
Drosophila male germline,
we used transgenic flies expressing the Salmonella phosphoinositide phosphatase SigD
under $\beta$2-tubulin promoter (referred to herein as \sigd{}),
which is uniquely expressed in spermatocytes
\citep{wei2008depletion, fabian2010phosphatidylinositol}.
We previously showed that expression of this construct
induced male sterility due in part to defects in flagellar biosynthesis \citep{wei2008depletion}.
We examined the localization of fluorescently-labelled centriolar/basal body protein
Ana1 (Cep295) \citep{goshima2007genes, blachon2009proximal}
and the TZ protein Cep290 \citep{basiri2014migrating}
in wild-type control and \sigd{} spermatocytes.
As shown in Figure 1A, nascent TZs marked with Cep290-GFP assemble normally in \sigd{} cells
but grow to much longer lengths compared to wild-type control cells.
TZ hyperelongation was a highly penetrant phenotype (\textgreater 90\%)
and showed high intra-cyst correlation.
This phenotype was not specific to Cep290;
all TZ markers examined showed hyperelongation, including
acetylated tubulin \citep{gottardo2013cilium},
Chibby \citep{enjolras2012drosophila} and
MKS \citep{slaats2015mks1, vieillard2016transition} proteins (Figure 1).
Hyperelongated transition zones persisted through meiosis.
Despite TZ hyperlongation, however, axonemes were able to elongate post-meiosis (Figure 1).
As we showed previously, the ultrastructure of these axonemes was frequently aberrant,
either lacking symmetry completely or containing triplet microtubules instead of doublets
\citep{wei2008depletion}.


\subsection{The type I PIP kinase Skittles localizes to centrioles and regulates TZ length}
Although numerous studies have shown that \PIP{} is the major substrate for SigD
when expressed in eukaryotic cells
\citep{terebiznik2002elimination, zhou2001salmonella, sengupta2013depletion},
SigD can dephosphorylate other PIPs \textit{in vitro}
\citep{norris1998sopb}.
To address whether TZ hyperelongation observed in \sigd{} represented
a physiologically relevant phenotype in response to decreased \PIP{},
we attempted to rescue it by
co-expressing fluorescently-tagged full-length Skittles (Sktl) with SigD
under the same promoter.
We found that Sktl expression was able to suppress TZ elongation in a cilium-autonomous
manner (Figure 2).
We also found that \textit{sktl} clones showed penetrant TZ hyperelongation when
marked with Unc-GFP, a ciliary marker \citep{baker2004mechanosensory}.

Since Sktl expression lead to cilium-autonomous rescue,
we investigated whether Sktl could localize to centrioles or cilia in the Drosophila germline.
Similar to the human PIPKI$\gamma$ isoform \citep{xu2014pipkigamma},
we found that Skittles localizes to basal bodies in the Drosophila male germline.
As shown in Figure 2, Drosophila PIP5Ks arose from the recent duplication of
the ancestral PIP5K gene, and are not orthologous to specific vertebrate PIP5K isoforms.
Skittles has diverged more than the paralogous PIP5K59B gene, and seems
to be functionally related to PIPKI$\gamma$ in its roles at cilia.
However, unlike the human PIPKI$\gamma$, which functions to license TZ assembly
by promoting CP110 uncapping from basal bodies \citep{xu2016phosphatidylinositol},
our results suggest that Skittles functions in
regulating TZ length in the Drosophila male germline
but is dispensable for its assembly.
The localization of Cp110 is unaffected in \sigd{} (supplementary figure 1).

\subsection{Hyperelongated transition zones show signs of functional defects}
While shorter spermatocyte TZs have been observed in Drosophila mutants of TZ proteins
\citep{vieillard2016transition, pratt2016drosophila},
no genetic manipulation has been reported to induce hyperelongated TZs. 
We sought to examine whether TZ hyperelongation due to SigD expression
affected TZ function.
During germ cell elongation following meiosis, the TZ detaches from
the basal body and migrates along the growing axoneme, maintaining a ciliary compartment
at the distal-most ~5$\mu$m where tubulin is presumably incorporated into the axoneme.
We found that
TZs that assembled in \sigd{} cells are non-migratory despite the growth of the
axoneme and the cell membrane, as shown by Unc and Cep290 localization.
We previously described ``comet-shaped'' localization of Unc-GFP in these cells,
and we find that these hyperelongated Unc signals persist into cellular elongation.

Another phenotype that is consistent with this hypothesis is
the observation of the centriolar protein Ana1 localizing to the distal tips of
hyperelongated TZs.
TZ-distal Ana1 localization is observed at a relatively low penetrance
and high intra-cyst correlation.
This is possibly related to stochastic SigD expression in cysts.
We found that treatment of germ cells with Paclitaxel as in \citep{riparbelli2013unique}
could increase the penetrance of this phenotype.
Paclitaxel treatment stabilizes microtubules and is thought to disrupt TZ maturation
by interrupting microtubule rearrangements, and similarly induces much longer cilia.
Paclitaxel treatment alone did not induce this phenotype in wild-type controls in our hands
at varying concentrations (binomial probability < 0.001 with 5\% penetrance),
suggesting that loss of \PIP{} acts upstream of or prior to microtubule rearrangement.
Although we can not show a definitive reason for this, we expect that this is due to
the axoneme maintaining some centriolar fate.
We had previously shown the presence of microtubule triplets in axonemes assembled in \sigd{}
cells.
These are also observed in Taxol-treated cells \citep{riparbelli2013unique}.
It is possible that the TZ defect allows C-tubules and some centriolar proteins
to migrate with it to the distal end.
We did not observe this behaviour with Asterless,
a migratory protein that localizes to the outer
ring of centrioles (Figure 3).

\subsection{The \textit{onion rings} mutant decouples defects found in cells with reduced levels of \PIP{}}
A heterozygous mutation in the human \textit{EXOC8},
the human ortholog of yeast Exo84, is linked to
Joubert syndrome \citep{dixon2012exome}.
We previously showed that male flies homozygous for the \textit{onion rings}
(\textit{onr}) mutant of Drosophila Exo84 were sterile, with phenotypes
similar to \sigd{} \citep{wei2008depletion}.
Exo84 is a component of the octameric exocyst complex, which binds
\PIP{} and regulates membrane trafficking at the plasma membrane.
We found that the \textit{onr} mutant did not display hyperelongated TZs (Figure 5).

Since exocyst regulates membrane trafficking and cilium formation,
we examined whether cilium-associated membrane dynamics were affected in \sigd{}
or \textit{onr} mutants similar to \textit{dila; cby} mutants
\citep{vieillard2016transition}.
We found that while TZs in \sigd{} and \textit{onr} cells anchored properly
to the plasma membrane, they were unable to either establish or maintain
membrane connections, and were rendered cytoplasmic (Figure 5) similar to
\textit{dila; cby} mutants.
Although we were not able to determine the localization of the Exo84 due to the
lack of available antibodies,
we found that Sec6 localized to the plasma membrane as expected.
Our results suggest that the exocyst may act downstream of \PIP{} to
enhance cilium-membrane associations, and that TZ hyperelongation and loss of
membrane-cilium association are genetically separable phenotypes.


\section{Discussion}

In this study, we show that the lipid \PIP{} is essential for proper
transition zone maturation in the male Drosophila germline.
Reduction of cellular \PIP{} by expression of the bacterial PIP
phosphatase SigD or inactivation of the type I PIP kinase Skittles
causes developing TZs to elongate excessively.


We had previously shown that reduction of \PIP{} by SigD expression
affected flagellar biogenesis and axoneme structure
\citep{wei2008depletion}.
We also showed aberrant ``comet-shaped'' localization of Unc that
failed to assemble TZ-associated puncta in some cases.
Here, we show that this occurs prior to meiosis,
is not an artefact due to cytokinesis defects
and do not represent an inability of the axoneme to grow.
Rather, 



It has previously been suggested that LCCS3 might represent a ciliopathy.
PIPKI$\gamma$ is linked to LCCS3 with many features reminiscent
of ciliopathies, such as micrognathia, and a role for the LCCS1 protein
Gle3 in cilia.
In addition, all three human PIPKI isoforms are highly expressed in testes
and PIPKI$\alpha$ and PIPKI$\beta$ are important for fertility in mice
\citep{hasegawa2012phosphatidylinositol}.




\section{Methods}
\subsection{Transgenic flies}
Drosophila stocks were cultured on cornmeal molasses agar medium at 25$^{\circ}$C
and 50\% humidity.
Stocks expressing $\beta$\textit{2tub}::SigD and $\beta$\textit{2tub}::YFP-Sktl
were described previously \citep{wei2008depletion, fabian2010phosphatidylinositol}.
$\beta$\textit{2tub}::mCherry-Sktl\textsuperscript{kinase-dead}
was generated by Gordon Polevoy.
Ana1-tdTomato (chromosome 2) and Cep290-GFP (chromosome 3) were provided
by T. Avidor-Reiss \citep{basiri2014migrating}.
Unc-GFP was originally provided by M. Kernan \citep{baker2004mechanosensory}.
Stocks expressing GFP-tagged Chibby, B9d1, B9d2, Cc2d2a and Mks1 were generously
provided by B. Durand \citep{enjolras2012drosophila, vieillard2016transition}.
The \textit{onr} mutant was described previously \citep{giansanti2015exocyst}.
Stocks for generating \textit{sktl}\textsuperscript{2.3} clones were originally provided by
A. Guichet \citep{gervais2008pip5k}.
\textit{w}\textsuperscript{1118} was used as the wild-type control.

\subsection{Antibodies}
The following primary antibodies were used for immunofluorescence
at the indicated concentrations - 
rabbit $\alpha$-Tulp (gift from W. McGinnis \citep{ronshaugen2002structure}), 1:500; 
chicken $\alpha$-GFP IgY (abcam) 1:1000;
rat $\alpha$-RFP IgG (5F8, ChromoTek), 1:1000;
rabbit $\alpha$-Sktl (courtesy of P. Raghu), 1:500;
Secondary antibodies were Alexa 488- and Alexa 568-conjugated
$\alpha$-mouse, $\alpha$-rabbit and $\alpha$-chicken
IgG (Molecular Probes) at 1:1000.
DAPI was used to label chromatin at 1:1000.

\subsection{Fluorescence microscopy}
For live imaging, testes were dissected in phosphate bufferd saline (PBS),
transferred to a polylysine-coated glass slide in a drop of PBS,
ruptured using a syringe needle and
squashed under a glass coverslip using Kimwipes.
The edges of the coverslip were sealed with nail polish
and the specimen was visualized immediately using a fluorescence microscope.

For immunochemistry, testes were dissected in PBS,
transferred to a polylysine-coated gladd slide in a drop of PBS,
ruptured with a needle, squashed and frozen in liquid nitrogen for 5 minutes.
The slides were transferred to ice-cold methanol for 5-10 minutes.
For staining with the $\alpha$-skittles antibody, pre-fixation in methanol
was omitted to preserve cellular membranes.
Samples were fixed in 4\% paraformaldehyde in PBS,
permeabilized and blocked in PBS with 0.1\% Triton-X and 0.3\% bovine
serum albumin and incubated with primary antibodies overnight at 4$^{\circ}$C.
followed by three 5-minute washes with PBS and 1 hour incubation
with secondary antibodies and three 5-minute washes with PBS.
Samples were mounted in Dako (Agilent) and imaged with Leica widefield microscope
or Nikon A1R scanning confocal microscope (SickKids imaging facility).

\subsection{Statistical methods}
Statistical analysis and graphing was performed using R.
Statistical tests for ``absence of phenotype'' were computed under a
hypothesis assuming binomial likelihood.
All $t$-tests were unpaired and two-sided.
A significance level of 0.01 was fixed \textit{a priori} for all analysis.

\section{Author contributions}
A.G. and J.A.B. conceived the project.
A.G. performed all the experiments and wrote the manuscript.
J.A.B. supervised the project.


\section{Acknowledgements}
We thank Brian Ciruna for input on results.

\bibliography{bibttc}

\end{document}
