\documentclass[12pt, twoside, letterpaper]{article}
\usepackage[margin=1in]{geometry}
\usepackage[parfill]{parskip}
\usepackage{graphicx}
\usepackage[utf8]{inputenc}
\usepackage[T1]{fontenc}
\usepackage{authblk}
\usepackage{charter}
\usepackage[charter]{mathdesign}
\usepackage[numbers]{natbib}
\usepackage{enumitem}
\usepackage{hyperref}
\usepackage{titlesec}
\usepackage{lineno}
\usepackage{setspace}
\graphicspath{{../figures/}}
\bibliographystyle{unsrtcurrbiol}

% title font size
\titleformat*{\section}{\large\bfseries}
\titleformat*{\subsection}{\normalsize\bfseries}

% line spacing
% \renewcommand{\baselinestretch}{1.1}

% newcommand
\newcommand{\PIP}{PIP\textsubscript{2}}
\newcommand{\sigd}{$\beta$\textit{2t}-SigD}
\newcommand{\sktl}{\textit{sktl}}

% renewcommand
\renewcommand*{\Affilfont}{\footnotesize \normalfont}
\renewcommand*{\Authfont}{\normalsize}


\title{\vspace{-1cm} \bfseries\large Phosphatidylinositol 4,5-bisphosphate regulates cilium transition zone maturation in \textit{Drosophila melanogaster}}
\author[1,2]{Alind Gupta \thanks{\url{alind.gupta@mail.utoronto.ca}}}
\author[1,2]{Julie A. Brill \thanks{\url{julie.brill@sickkids.ca} (Corresponding author)}}

\affil[1]{\small Department of Molecular Genetics, University of Toronto, Toronto, ON M5S 1A8, Canada}
\affil[2]{\small Program in Cell Biology, Hospital for Sick Children, Toronto, ON M5G 1X8, Canada}
\date{\small \today}


\begin{document}

\maketitle

\begin{doublespacing}
  \begin{linenumbers}
    
    \section{Summary}
    Cilia are sensory organelles that are essential for human development
    and physiology \citep{satir2010primary, marshall2006cilia}.
    They function primarily by relaying extracellular cues to cells
    \citep{satir2010primary},
    but can also act as motile propellers in certain cell types
    \citep{bloodgood2010sensory}.
    A large number of genetic disorders linked to cilium dysfunction are
    associated with proteins that localize to the cilium transition zone (TZ),
    a structure at the base of cilia that regulates trafficking in and out
    of the cilium \citep{reiter2012base, szymanska2012transition}.
    Although substantial effort has been undertaken
    to elucidate TZ proteins and their functions in cilium assembly and function,
    processes underlying maturation of TZs are not well understood.
    Here, we report a role for the membrane lipid
    phosphatidylinositol 4,5-bisphosphate
    (\PIP{}) in TZ maturation and length regulation
    in the \textit{Drosophila melanogaster} male germline.
    We show that reduction of cellular \PIP{} levels
    by expression of the Salmonella phosphoinositide phosphatase SigD,
    or mutation of
    the type I phosphatidylinositol phosphate kinase (PIPKI) Skittles (Sktl),
    induces the formation of longer than normal TZs.
    These hyperelongated TZs exhibit signs of functional defects,
    such as the inability
    to be migratory \citep{basiri2014migrating} and loss of membrane tethering.
    We also report that the \textit{onion rings} (\textit{onr}) allele
    of the Drosophila Exo84
    decouples TZ hyperelongation from loss of cilium-plasma membrane tethering.
    Our results reveal a requirement for \PIP{} in supporting ciliogenesis
    by promoting proper TZ maturation.
    
    \section{Results}
    Cilia are sensory organelles that are present on almost all cells in the human body
    \citep{satir2010primary, marshall2006cilia}.
    They function in signal reception and transduction in response to extracellular cues,
    but also have roles in cellular and extracellular fluid motility
    \citep{brooks2014multiciliated, marshall2006cilia},
    for example, as the sperm flagellum.
    Consistent with their importance in cell signalling,
    defects in cilium formation (i.e. ciliogenesis) are associated with genetic disorders
    known collectively as ciliopathies, which can display
    neurological defects \citep{valente2014primary},
    skeletal abnormalities \citep{hammarsjo2017novel},
    and infertility \citep{inaba2016sperm} in addition to other phenotypes
    \citep{waters2011ciliopathies}.
    Many ciliopathies are associated with mutations in proteins that localize
    to the transition zone (TZ), the proximal-most region of the cilium which
    functions as a diffusion barrier and regulates the
    bi-directional transport of protein cargo at the base of the cilium
    \citep{reiter2012base, szymanska2012transition}.
    The TZ is essential for ciliogenesis and function.
    For example, the conserved TZ protein CEP290 is mutated in at least
    six different ciliopathies that lie on a spectrum of clinical features and severities,
    and is important for cilium formation
    and function in both humans \citep{shimada2017vitro, stowe2012centriolar}
    and Drosophila \citep{basiri2014migrating}.
    Although the protein composition of TZs has been investigated in various
    studies ranging from single gene analysis to proteomics,
    the process of maturation of the TZ, through which it is converted from a
    immature form to one competent at supporting
    cilium assembly, is relatively understudied.

    Ciliogenesis begins with the assembly of a nascent TZ at the tips of
    the basal body \citep{reiter2012base}.
    This nascent TZ undergoes maturation, during which its structure and protein
    composition changes, allowing for the
    establishment of a compartmentalized ciliary space bounded by the ciliary membrane
    and the TZ in which axoneme assembly and signalling can occur.
    The axoneme is a microtubule-based structure that forms the core of all cilia and flagella.
    In the Drosophila male germline, nascent TZs first assemble on basal bodies
    during early G2 phase in spermatocytes \citep{riparbelli2012assembly}.
    This is followed by anchoring of cilia to the plasma membrane,
    rearrangment of microtubules within the TZ \citep{gottardo2013cilium},
    and establishment of a
    ciliary membrane that will persist through meiosis
    \citep{riparbelli2012assembly} (Figure 1A).
    TZ maturation has been described in
    \textit{Paramecium} \citep{aubusson2015transition}
    \textit{Caenorrhabditis elegans} \citep{serwas2017centrioles} and
    \textit{Drosophila melanogaster} \citep{gottardo2013cilium},
    and is most readily observed by an increase in TZ length
    in the Drosophila male germline.

    We previously showed that the membrane lipid phosphatidylinositol 4,5-bisphosphate
    (\PIP{}) is essential for the formation of the sperm flagellum in the Drosophila
    male germline \citep{wei2008depletion, fabian2010phosphatidylinositol}.
    \PIP{} is one of seven different phosphoinositides (PIPs) present in eukaryotes.
    It localizes primarily to the plasma membrane, where it is essential for
    modulation of actin cytoskeleton and vesicle trafficking amongst other things
    \citep{balla2013phosphoinositides}.
    \PIP{} has been linked to cilium function in many recent studies.
    The ciliary membrane has been shown to contain very little \PIP{} due to the action
    of the cilium resident lipid phosphatase INPP5E \citep{nakatsu2015phosphoinositide}.
    Inactivation of INPP5E leads to a build up of intraciliary \PIP{}
    which disrupts transport of certain signalling proteins in both vertebrates
    \citep{chavez2015modulation, garcia2015phosphoinositides, conduit2017compartmentalized}.
    and Drosophila \citep{park2015ciliary}.
    We sought to investigate the defects in axoneme assembly in Drosophila male germ cells
    with reduced \PIP{} 
    that we reported previously in light of the current understanding of the roles
    for \PIP{} in cilia as a modulator of cilium function.

    %
    %   Results - 1
    %
    \subsection{\PIP{} is essential for transition zone maturation}
    To investigate how the reduction of cellular \PIP{} affects ciliogenesis in the
    Drosophila male germline,
    we used transgenic flies expressing the Salmonella phosphoinositide phosphatase SigD
    under the spermatocyte-specific $\beta$2-tubulin promoter (referred to herein as \sigd{}),
    \citep{wei2008depletion, fabian2010phosphatidylinositol}.
    SigD dephosphorylates \PIP{} \textit{in vivo},
    and we previously showed that \sigd{} induces male sterility
    due in part to defects in flagellar biosynthesis \citep{wei2008depletion}.
    To examine whether these were caused by aberrant TZ function,
    we examined the localization of fluorescently-tagged
    centriolar/basal body protein
    Ana1 (homolog of CEP295) \citep{goshima2007genes, blachon2009proximal}
    and the conserved TZ protein Cep290 \citep{basiri2014migrating}
    during early steps of cilium assembly in spermatocytes in both
    wild-type controls and \sigd{}.
    As shown in Figure 1,
    while Cep290 looks similar in control and \sigd{} in early G2 phase
    when TZs are still immature, 
    it grows to significantly longer lengths in \sigd{}
    compared to wild-type controls by the end of the G2 phase following
    the period of TZ maturation in spermatocytes.
    In contrast to the \textit{cep290} mutant, which contains
    longer than normal basal bodies \citep{basiri2014migrating},
    Ana1 length was not affected in \sigd{},
    and we did not observe a strong
    correlation between Cep290 and Ana1 lengths (Figure 1C).
    TZ proteins Chibby (Cby) \citep{enjolras2012drosophila} and
    Mks1 \citep{vieillard2016transition}
    showed hyperelongation in \sigd{} (Figure 1D),
    suggesting that this phenotype is not specific to Cep290 but is shared
    by other members of the Drosophila TZ.
    TZ hyperelongation was a highly penetrant phenotype (\textgreater 70\%
    in spermatocytes at late G2 phase)
    and showed high intra-cyst correlation (\textgreater 0.95) suggestive
    of a dosage-based response to a shared cellular factor,
    presumably SigD.
    Despite the persistence of hyperelongated TZs through meiosis,
    axonemes were able to
    elongate normally in post-meiotic cells (Figure 1E).
    However, as we showed previously, the ultrastructure of these axonemes
    was frequently aberrant,
    either lacking symmetry completely or containing triplet microtubules
    in addition to the usual doublets \citep{wei2008depletion}.

    %
    %   Results - 2
    %
    \subsection{The type I PIP kinase (PIPKI) Skittles (Sktl) regulates TZ length}
    While \PIP{} is the major substrate for SigD in eukaryotic cells \textit{in vivo}
    \citep{terebiznik2002elimination, zhou2001salmonella, sengupta2013depletion},
    SigD can also dephosphorylate other PIPs \textit{in vitro}
    \citep{norris1998sopb}.
    To address whether TZ hyperelongation observed in \sigd{} represented a
    physiologically relevant phenotype in response to decreased \PIP{},
    we attempted to rescue it by
    co-expressing SigD with
    fluorescently-tagged full-length Skittles (Sktl)
    using the same $\beta$2-tubulin promoter.
    We found that Sktl expression was able to suppress TZ hyperelongation
    in a cilium-autonomous manner (Figure 2A and B).
    We also found that \textit{sktl} clones made
    using Flp/FRT-based mitotic recombination and expressing the
    basal body and TZ marker Unc
    \citep{baker2004mechanosensory, wei2008depletion}
    exhibited TZ hyperelongation (Figure 2C),
    suggesting that Sktl is important for TZ maturation.

   % Since Sktl expression suppressed TZ hyperelongation,
   % we investigated whether Sktl could localize to centrioles or cilia
   % in the Drosophila germline.
    Xu \textit{et al}. showed the vertebrate PIPKI$\gamma$ to be
    important for cilium formation in cultured cell lines
    \citep{xu2014pipkigamma}.
    As shown in Figure 2D, Drosophila PIPKIs arose from the recent duplication of
    the ancestral PIPKI gene,
    and are not orthologous to specific vertebrate PIPKI isoforms.
    Sktl has diverged more than its paralog PIP5K59B, and seems
    to be functionally related to PIPKI$\gamma$ in its roles at cilia,
    as well as the \textit{C. elegans} PPK-1 (See supplementary data in \citep{xu2014pipkigamma}).
    However, unlike the human PIPKI$\gamma$, which functions to license TZ assembly
    by promoting CP110 removal from basal bodies \citep{xu2016phosphatidylinositol},
    our results suggest that Sktl functions in
    regulating TZ length in the Drosophila male germline
    but not TZ assembly.
    Cp110 is dispensable for cilium formation in Drosophila \citep{franz2013cp110}
    and we found its localization to be unaffected in \sigd{} (Supplementary figure 1).
    
    %
    %   Results - 3
    %
    \subsection{Hyperelongated transition zones exhibit signs of functional defects}
    We sought to examine whether TZ hyperelongation due to SigD expression
    affected TZ function.
    During cell elongation following meiosis, the TZ detaches from
    the basal body and migrates along the growing axoneme, maintaining a ciliary compartment
    at the distal-most {\raise.17ex\hbox{$\scriptstyle\sim$}}5$\mu$m where tubulin
    is presumably incorporated into the axoneme
    \citep{basiri2014migrating, fabian2012drosophila}.
    We found that
    TZs that assembled in \sigd{} cells are non-migratory despite the growth of the
    axoneme and the cell membrane, as shown by Unc and Cep290 localization
    (Figures 3A and 3B).
    We previously described ``comet-shaped'' localization of Unc-GFP in these cells,
    and we find that these hyperelongated Unc signals persist into cellular elongation
    despite elongation of the axoneme (Figure 3A, bottommost panel).

    In Drosophila and humans, basal bodies consist of microtubule triplets
    \citep{jana2016drosophila, lattao2017centrioles}
    while the axoneme contains microtubule doublets due to the obstruction of
    the C-tubule at the TZ \citep{gottardo2013cilium}.
    Consistent with a defect in the TZ barrier and the presence of
    microtubule triplets in axonemes in \sigd{} \citep{wei2008depletion},
    we observed that a subset of cilia (<5\%) in SigD-expressing cells
    contained puncta of Ana1, a core centriolar protein, at the tips
    of TZs marked with Cep290 (Figure 3C).
    We found that treatment of germ cells with Taxol as in \citep{riparbelli2013unique}
    could increase the penetrance of this phenotype (Figure 3D).
    Taxol treatment stabilizes microtubules and is thought to disrupt TZ maturation
    by inhibiting microtubule rearrangements \citep{riparbelli2012assembly}.
    Taxol treatment alone did not induce this phenotype in wild-type controls in our hands
    at varying concentrations ($p$ <0.01 at 5\% penetrance).
    Triplet microtubules within the axoneme are also observed in Taxol-treated cells
    \citep{riparbelli2013unique}, further supporting a relationship between
    \PIP{} and microtubule reorganization in TZ maturation.
    Asterless (homolog of CEP152), another centriolar protein,
    did not localize to TZ-distal puncta in \sigd{}
    \citep{dzhindzhev2010asterless, blachon2008drosophila} ($p$ <0.01)
    suggesting that these TZ-distal
    sites are not fully centriolar in protein composition.

    %
    %   Results - 4
    %
    \subsection{The \textit{onion rings} mutant decouples defects found in cells with reduced levels of \PIP{}}
    We previously showed that male flies homozygous for the \textit{onion rings}
    (\textit{onr}) mutant of Drosophila Exo84 were sterile, and had phenotypes
    similar to \sigd{} such as defects in cell polarity \citep{wei2008depletion}.
    Exo84 is a component of the octameric exocyst complex, which binds
    \PIP{} and regulates membrane trafficking at the plasma membrane.
    To investigate whether defects in TZ hyperelongation could be explained by
    defective Exo84 function, we examined TZs in \textit{onr} mutants.
    We found that the \textit{onr} mutant did not display hyperelongated TZs (Figure 5C),
    suggesting that Exo84 is dispensable for TZ maturation.

    Due to the involvement of exocyst in membrane trafficking at the plasma membrane,
    we examined whether cilium-associated membrane dynamics were affected in \sigd{}
    or \textit{onr} mutants similar to \textit{dila; cby} mutants
    \citep{vieillard2016transition}.
    We found that while TZs in \sigd{} and \textit{onr} cells anchored properly
    to the plasma membrane, they were unable to either establish or maintain
    membrane connections, and were rendered cytoplasmic (Figure 5A and B) similar to
    \textit{dila; cby} mutants.
    Although we were not able to determine the localization of Exo84 due to the
    lack of available reagents,
    we found that GFP-tagged Exo70 localized to centrioles (Figure 4D).
    Exo70 can directly bind to \PIP{} and mediates the recruitment of
    the exocyst complex to the plasma membrane \citep{he2007exo70}.
    Our results suggest that the exocyst may act downstream of \PIP{} to
    enhance cilium-membrane associations, and that TZ hyperelongation and loss of
    membrane-cilium association are genetically separable phenotypes.

    \section{Discussion}
    The process of maturation of a TZ from a nascent form to a fully functional one,
    leading ultimately to axoneme assembly and ciliary signalling,
    requires orchestration of various proteins and cellular pathways
    \citep{reiter2012base}.
    In the Drosophila male germline, TZ maturation occurs during the
    extended G2 phase in spermatocytes prior to meiosis, and can be readily
    observed by an increase in TZ length using fluorescence microscopy.
    Our results suggest that normal execution of this process
    requires \PIP{}, and that depletion of \PIP{} induces TZs to
    grow longer than normal.
    Regulation of TZ length in the Drosophila male germline has been reported
    previously.
    For example, \textit{mks1} mutants have shorter TZs \citep{pratt2016drosophila},
    while
    \textit{cby; dila} and \textit{cby} mutants display hyperelongated ones
    \citep{enjolras2012drosophila,vieillard2016transition}.
    Since both Cby and Mks1 are hyperelongated in \sigd{} cells,
    \PIP{} seems to be intimately involved in this process as well but it does
    so independantly of Cby and Mks1.

    The majority of \PIP{} at the plasma membrane is produced by PIPKIs
    \citep{balla2013phosphoinositides}
    and these seem to have a conserved role at cilia.
    In this study, we showed that mutation of the PIPKI Skittles (Sktl)
    induced hyperelongated TZs, similar to SigD expression.
    We were able rescue TZ hyperelongation by expression of full-length Sktl to
    a large degree.
    Interestingly, some cells showed cilium-autonomous suppression of lengths.
    The development of the Drosophila male germline occurs in syncytial ``cysts''
    of cells which share the same cytoplasm, and
    our results suggest that Skittles might function \textit{in situ}
    to regulate TZ maturation in these cells.
    In humans, \textit{PIPKI$\gamma$} is linked to lethal congenital contractural
    syndrome type 3 (LCCS3), which has been suggested to represent a ciliopathy.
    The recent discovery of the role of LCCS1-associated GLE1 protein in cilium
    function \citep{jao2017role} further supports this hypothesis.
    Thus, PIPKIs might represent novel ciliopathy-associated genes
    or genetic modifiers of disease.
    
    The gene \textit{EXOC8}, encoding the human Exo84, has also
    been linked to Joubert syndrome, a ciliopathy
    \citep{dixon2012exome}.
    Members of the exocyst complex, such as Sec10 and Sec6, have been shown
    to be important for cilium formation in cultured cell lines and zebrafish
    \citep{zuo2009exocyst, lobo2017exocyst, seixas2016arl13b}.
    The exocyst is known to bind to \PIP{}
    \citep{he2007exo70, zhang2008membrane}, and we previously showed that
    the \textit{onr} allele of Drosophila \textit{Exo84} phenocopied
    male germline defects observed in \sigd{} \citep{fabian2010phosphatidylinositol}.
    In this paper, we show that the \textit{onr} allele of the Drosophila Exo84
    phenocopies the loss of cilium-membrane contacts in \sigd{},
    but not TZ hyperelongation.
    Proper TZ function is required for membrane tethering, as evidenced by
    \textit{dila; cby} mutants, which contain abnormal TZs
    and show similar cytoplasmic cilia similar as \sigd{} \citep{vieillard2016transition}.
    Our results suggest that the exocyst is required for this process,
    possibly downstream of \PIP{}.

    \section{Methods}
    \subsection{Transgenic flies}
    Drosophila stocks were cultured on cornmeal molasses agar medium at 25$^{\circ}$C
    and 50\% humidity.
    Stocks expressing $\beta$\textit{2t}::\textit{Styp}\textbackslash{SigD} (chromosome \textit{3}),
    $\beta$\textit{2t}::YFP-Sktl (chromosome \textit{2}) and
    Exo70-GFP
    were described previously \citep{wei2008depletion, fabian2010phosphatidylinositol}.
    $\beta$\textit{2t}::mCherry-Sktl was made as in \citep{wei2008depletion},
    and a \textit{P}-element insertion on chromosome \textit{2} was used for rescue using the low-level
    expression vector \textit{tv3} \citep{wong2005pip2}.
    Ana1-tdTomato (chromosome \textit{2}) and Cep290-GFP (chromosome \textit{3}) were provided
    by T. Avidor-Reiss \citep{basiri2014migrating}.
    \textit{Sp}/\textit{CyO}; Unc-GFP was originally provided by M. Kernan \citep{baker2004mechanosensory}.
    Stocks expressing GFP-tagged Chibby and Mks1 were
    provided by B. Durand \citep{enjolras2012drosophila, vieillard2016transition}.
    The \textit{onr} mutant was described previously \citep{giansanti2015exocyst}.
    Stocks for generating \textit{sktl}\textsuperscript{2.3} clones were originally provided by
    A. Guichet \citep{gervais2008pip5k}.
    \textit{w}\textsuperscript{1118} was used as the wild-type control.

    \subsection{Antibodies}
    The following primary antibodies were used for immunofluorescence
    at the indicated concentrations -
    chicken anti-GFP IgY (abcam) 1:1000;
    rat anti-RFP IgG (5F8, ChromoTek), 1:1000;
    rabbit anti-Cp110 (courtesy of J. Raff), 1:500;
    rabbit anti-Centrin {Placeholder for origin}, 1:500;
    mouse anti-acetylated $\alpha$-tubulin 6-11-B (Sigma-Aldrich), 1:1000.
    Secondary antibodies were Alexa 488- and Alexa 568-conjugated
    anti-mouse, anti-rabbit and anti-chicken
    IgG (Molecular Probes) at 1:1000.
    DAPI at 1:1000 was used to label chromatin.

    \subsection{Fluorescence microscopy}
    For live imaging, testes were dissected in phosphate bufferd saline (PBS).
    If staining for chromatin, intact testes were incubated in PBS with
    Hoechst (1:5000) for 5 minutes at this point.
    Testes were transferred to a polylysine-coated glass slide in a drop of PBS,
    ruptured using a syringe needle and
    squashed under a glass coverslip using Kimwipes.
    The edges of the coverslip were sealed with nail polish
    and the specimen was visualized immediately using an epifluorescence microscope.

    For Taxol treatments, testes from larvae or pupae expressing
    Ana1-tdTomato; Cep290-GFP were
    dissected into Shields and Sang M3 medium (Sigma-Aldrich) supplemented
    with a predefined
    concentration of Taxol and incubated overnight in a humidified sterile
    chamber in the dark at room temperature. They were then squashed
    in PBS and imaged live.

    For CellMask staining, cells were spilled from testes in M3 medium onto
    a sterilized glass-bottom dish that was pre-treated with sterile polylysine solution
    to enable cells to adhere.
    CellMask Deep Red (Invitrogen) solution (20 $\mu$g/mL) was added to the medium dropwise
    immediately before visualization under a confocal microscope.

    For immunochemistry, testes were dissected in PBS,
    transferred to a polylysine-coated glass slide in a drop of PBS,
    ruptured with a needle, squashed and frozen in liquid nitrogen for 5 minutes.
    The slides were transferred to ice-cold methanol for 5-10 minutes for pre-fixation.
    For staining with the anti-Sktl antibody, this pre-fixation step
    was omitted to preserve cellular membranes.
    Samples were fixed in 4\% paraformaldehyde in PBS,
    permeabilized and blocked in PBS with 0.1\% Triton-X and 0.3\% bovine
    serum albumin and incubated with primary antibodies overnight at 4$^{\circ}$C.
    followed by three 5-minute washes with PBS and 1 hour incubation
    with secondary antibodies and three 5-minute washes with PBS.
    Samples were mounted in Dako (Agilent) and imaged with
    a Leica widefield microscope
    or Nikon A1R scanning confocal microscope (SickKids imaging facility).

    \subsection{Statistical methods}
    Statistical analysis and graphing was performed using R (version 3.4)
    \citep{r}.
    A Gaussian jitter was applied when plotting
    results in Figures 1 and 2 for clearer visualization of trends,
    but raw data was used for all analysis.
    Statistical tests for ``absence of phenotype'' were computed under a
    hypothesis assuming binomial likelihood (i.e. assuming a fixed probability of the
    phenotype occurring).
    All $t$-tests were unpaired and two-sided.
    A significance level of 0.01 was fixed \textit{a priori} for all classical analysis.
    All raw data and code for analysis and plotting can be found online
    at \url{http://www.github.com/alindgupta/germline-paper/}.

    \subsection{Phylogenetic analysis}
    Candidate orthologs of Skittles or PIP5K9B were queried
    from Inparanoid and FlyBase (version 8.0).
    Poorly annotated protein sequences were confirmed
    to encode type I phosphatidylinositol phosphate
    kinases using reciprocal BLAST search.
    Phylogeny.fr (\url{http://www.phylogeny.fr}) was used for
    phylogenetic reconstruction with T-Coffee for multiple alignment
    and MrBayes for tree construction.
    The output was converted to a vector image in Illustrator
    and colours were added for the purpose of illustration.

    \section{Supplemental information}
    Includes two figures. (Pending attachment)
    
    \section{Acknowledgements}
    We are grateful to Brian Ciruna for insightful comments on the project,
    and B{\'e}n{\'e}dicte Durand for providing (at the time of experiments)
    unpublished fly stocks.
    This work was supported by a grant from the Canadian Institute for
    Health and Research to J.A.B (Placeholder for Grant number).
    A.G. was supported by an Open Fellowship and Ontario Graduate Scholarship.
    
    \section{Author contributions}
    A.G. and J.A.B. conceived the project.
    A.G. performed all the experiments and analysis, and wrote the manuscript.
    J.A.B. supervised the project.
    
    \section{Declaration of interests}
    The authors declare no competing interests.

    \section{Figure legends}
    
    \textbf{Figure 1. SigD expression induces transition zone hyperelongation.}
    \begin{enumerate}[label={(\Alph*)}, nolistsep]
    \item Schematic diagram of the stages of ciliogenesis in the Drosophila male germline.
      Stages in brackets correspond to those in \citep{cenci1994chromatin}.
    \item \sigd{} expression induces Cep290 hyperelongation in late G2 cilia.
    \item Paired Ana1-Cep290 lengths in early and late G2 phase in spermatocytes.
    \item \sigd{} expression induces hyperelongation of Chibby and Mks1 in late G2 phase.
    \item TZ hyperelongation persists through meiosis but does not affect
      axoneme outgrowth.
    \end{enumerate}
    
    \textbf{Figure 2. Sktl is important for transition zone maturation.}
    \begin{enumerate}[label={(\Alph*)}, nolistsep]
    \item Expression of full-length Sktl suppresses \sigd{}-induced TZ hyperelongation.
    \item Quantification of Cep290 and Ana1 lengths from control, \sigd{} and Sktl; \sigd{} from (A).
    \item \textit{sktl}$^{2.3}$ clones exhibit TZ hyperelongation. Cilia are marked using
      Unc-GFP.
    \item Phylogenetic tree of PIP5Ks showing conservation of cilium-associated functions.
      The scale bar represents expected amino acid substitutions per site and branch support values are shown in red (a value of 1 indicates maximum support).
      Black arrows represent involvement in cilium-associated functions.
      Abbreviations: Cele (\textit{Caenorhabditis elegans}), Spur (\textit{Strongylocentrotus purpuratus}), Amel (\textit{Apis mellifera}), Aaeg (\textit{Aedes aegypti}), Dana (\textit{Drosophila ananassae}), Dmel (\textit{Drosophila melanogaster}), Hsap (\textit{Homo sapiens}), Mmus (\textit{Mus musculus}), Xtro (\textit{Xenopus tropicalis}), Cint (\textit{Ciona intestinalis}), Scer (\textit{Saccharomyces cerevisiae}).
    \end{enumerate}
    
    
    \textbf{Figure 3. Hyperelongated transition zones display functional defects.}
    \begin{enumerate}[label={(\Alph*)}, nolistsep]
    \item GFP-tagged Unc is unable to detach from the basal body and migrate in spermatids expressing \sigd{}.
    \item Cep290 is unable to detach and migrate from the basal body at the onset of axoneme assembly in \sigd{} spermatids.
    \item Structured illumination micrographs of \sigd{} cells show TZ-distal puncta of the centriolar protein Ana1.
    \item Treatment of \sigd{} cells with the microtubule stabilizing drug Taxol increases the penetrance of TZ-distal Ana1.
    \item Quantification of Cep290 lengths in Taxol-treated control and \sigd{} cells.
      % Inset shows distribution of regression coefficients (i.e. effects of Taxol and SigD on Cep290 length) from a fully Bayesian analysis.
    \end{enumerate}


    \textbf{Figure 4. The \textit{onion rings} (\textit{onr}) allele of \textit{d}Exo84 decouples TZ hyperlongation from loss of plasma membrane contacts.}
    \begin{enumerate}[label={(\Alph*)}, nolistsep]
    \item Cells expressing \sigd{} do not maintain plasma membrane-cilium tethering. The plasma membrane is marked with CellMask.
    \item \textit{onr} mutants do not maintain plasma membrane-cilium tethering.
    \item Cells mutant for \textit{onr} do not display hyperelongated acetylated tubulin signal at the cilium.
    \item GFP-tagged Exo70 localizes to basal bodies.
    \end{enumerate}
    
  \end{linenumbers}
\end{doublespacing}

\bibliography{bibttc}

\newpage

\section{Figures}

\textbf{Figure 1}
\begin{figure}[ht]
  \includegraphics[scale=0.9]{figure-1/figure1.png}
\end{figure}
\newpage

\textbf{Figure 2}
\begin{figure}[ht]
  \includegraphics[scale=0.9]{figure-2/figure2.png}
\end{figure}
\newpage

\textbf{Figure 3}
\begin{figure}[ht]
  \includegraphics[scale=0.8]{figure-3/figure3.png}
\end{figure}
\newpage

\textbf{Figure 4}
\begin{figure}[ht]
  \includegraphics[scale=0.9]{figure-4/figure4.png}
\end{figure}
\newpage

\textbf{Supplemental figures}
Cp110 localization (pending attachment).

\end{document}

%%% Local Variables:
%%% mode: latex
%%% TeX-master: t
%%% End
